\documentclass[fleqn,10pt]{wlscirep}
\title{Global learning, local Flow: subjective Flow report responds to proximal performance reference-level in a longitudinally-measured game-like task}

% insert here the call for the packages your document requires
\graphicspath{{./Figures/}}
% \usepackage[latin1]{inputenc}
\usepackage{amsmath}
\usepackage{amsfonts}
\usepackage{amssymb}
\usepackage{url}
\usepackage{xspace}
\usepackage{textcomp}
\usepackage{xcolor}
\usepackage{varwidth}
\usepackage{todonotes}

% please place your own definitions here and don't use \def but \newcommand{}{}
\newcommand{\hl}{\textcolor{red!60}}
\newcommand{\CCS}{\textsf{CogCarSim}\xspace}
\newcommand{\nicewidth}{0.8\textwidth}
\newcommand{\halfnicew}{0.4\textwidth}
\newcommand{\tapprx}{\raisebox{0.4ex}{\texttildelow}}


\author[1,2,*]{Benjamin Ultan Cowley}
\author[1]{Roosa Frantsi}%YES
\author[1,3]{Ville-Pekka Inkil\"{a}}%YES
\author[1]{Noora Lehtonen}%YES
\author[1]{Jussi Palom\"{a}ki}%YES
\author[1]{Pasi P\"{o}l\"{o}nen}
\author[1]{Tuisku Tammi}%YES
\author[1]{Kalle Toikka}%YES
\author[1]{Juha Veps\"{a}l\"{a}inen}%YES
\author[1]{Otto Lappi}
\affil[1]{Cognitive Science, Department of Digital Humanities, University of Helsinki, Helsinki, Finland}
\affil[2]{Cognitive Brain Research Unit, Department of Psychology and Logopedics, University of Helsinki,
Helsinki, Finland}
\affil[3]{Digitalization, Finnish Institute of Occupational Health, Helsinki, Finland}

\affil[*]{ben.cowley@helsinki.fi}

\affil[+]{these authors contributed equally to this work}

%\keywords{Keyword1, Keyword2, Keyword3}

\begin{abstract}
Example Abstract. Abstract must be under 200 words and not include subheadings or citations. Example Abstract. Abstract must be under 200 words and not include subheadings or citations. Example Abstract. Abstract must be under 200 words and not include subheadings or citations. Example Abstract. Abstract must be under 200 words and not include subheadings or citations. Example Abstract. Abstract must be under 200 words and not include subheadings or citations. Example Abstract. Abstract must be under 200 words and not include subheadings or citations. Example Abstract. Abstract must be under 200 words and not include subheadings or citations. Example Abstract. Abstract must be under 200 words and not include subheadings or citations.
\end{abstract}
\begin{document}

\flushbottom
\maketitle
\thispagestyle{empty}

\section*{Introduction}
{\sf NO \textbf{subheadings} are permitted. Temporary subheadings should be deleted before submission.}

% AIM
% In this report we examine how analysis of individual subjects can uncover patterns in the subjective Flow experience obtained during performance of a novel task.
The experiential state of Flow, described by the `theory of optimal experience' \cite{Nakamura2002}, is the phenomenal correlate of the act of performing very demanding tasks with high skill. Flow theory has been well-studied across several decades using an increasingly broad range of methods and settings \cite{Engeser2012intro}. New psychological models of Flow \cite{Keller2012} (updating the classic channel and quadrant/octant models) show the conceptual advance of Flow research, while the number of studies on the psychophysiology of Flow has also risen during recent years \cite{Peifer2012,Peifer2014,Wolf2015,Harmat2015,Labonte-LeMoyne2016}.

One area of current interest is the relationship between Flow and task-relevant skill acquisition, which brings {\it on-task} learning into interaction with Flow. Psychological theory of Flow has dealt with state and trait models \cite{Moneta2012}, but to deal with on-task learning, Flow must be considered as a dynamic process (or linked to a dynamic process). Prior work has proposed such an account using (neuro)cognitive models of information-processing: including Marr's biobehavioural model \cite{Marr2001}; this lead author's model, which integrated elements from information theory to neurobiology \cite{Cowley2008}; or more recently, a descriptive model proposed by {\v{S}}imle{\v{s}}a and colleagues \cite{Simlesa2018} using a combustion-engine metaphor. However, these models have not yet been empirically tested.

One issue for linking Flow to learning is that existing Flow theory is not intrinsically built to handle the difference between simple and non-simple tasks: e.g. washing dishes versus driving cars. If we allow that Flow can occur in all sorts of tasks, and the tasks will be characterised by the depth of learning available, then it follows that Flow-as-a-process depends on the task `depth'. To capture this, we use the term `high-performance cognition' (HPC) to describe Flow in feats of enactive cognition with high skill. In HPC-class tasks, skill acquisition is required and is non-trivial for any duration of learning, i.e. HPC tasks must have a shallow learning curve (learning is slow). Importantly, the skill level does not quickly peak, such as with simpler tasks like washing the dishes, where Flow might be obtained but does not strongly interact with learning.
\todo[author=BC,inline]{Some semantic clashing between shallow learning curves (slow learning) and deep tasks (non-trivial skills), which are the same thing.}

{\bf Nevertheless, the question of whether Flow is indeed a state, or a dynamic process, requires empirical testing}.

Flow-as-a-state leads us to expect that as skills improve and challenges are consequently raised, Flow self-report should increase over time. Consider a series of Flow-inducing sessions with  skill acquisition, e.g. in a driving simulator. In such a `HPC' task, learning implies skill increase, which implies challenge increase, which together imply Flow increase. This is true whether we follow the classic Flow quadrant/octant model \cite{Massimini1988}, or Keller and Landhausser's \cite[pp-56]{Keller2012} recent revision that describes Flow intensity as a function of perceived fit and subjective value.

In the octant model, if skills and challenges increase, the individual should feel further `north-east' of the median point where Flow bottoms out, and thus be more likely to report Flow and assign it greater intensity on a reporting scale. In the Flow intensity model \cite{Keller2012}, increased skills should increase both perceived fit of skills and task demands (because the task's deeper levels of challenge are uncovered precisely by learning the skills to meet them); {\it and} subjective value of the activity (because learning skills increases the learners investment in the task). In summary, for Flow-as-a-state the reference frame is fixed, leading to the prediction of increased Flow with skill acquisition.
\todo[author=BC,inline]{The above paragraph goes a bit off the deep end. Is it needed? If so, perhaps goes better to Discussion.}

However if Flow is a dynamic process it must be bounded, and then the natural homeostatic quality of a bounded dynamic process will tend to maintain Flow self-report at a level proportionate to the temporally local self-concept of skills and challenges.

Here, we aim to provide empirical data to critically examine Flow models in the context of the larger question of Flow and learning, with particular focus on the Flow intensity model \cite{Keller2012} (as the octant/quadrant models have been well-criticised \cite{Moneta2012}).

We present an empirical study of Flow that probes the interaction of learning and optimal experience {\it in a longitudinal-task setting which allows for complete control of task variables and analysis of behaviour with high temporal precision}.

Learning relates to predisposition to learn in the context, which is governed by psychological factors (motivation, prior experience), but also physio-/neurological (e.g. dopamine is related to attentional blink \cite{Slagter2012}). Spontaneous blink rate has been suggested as an externally-measureable index of striatal dopamine level, and may thus hold explanatory power with respect to individual variability in learning among participants. Thus, our experimental setting also incorporates eye tracking.

We report results that show how Flow assessment relates to the local reference frame provided by learning curves; with evidence that the learning curve across sessions is predicted by subjects' spontaneous blink rate.


\subsection{X Background X}
Flow is a combination of distinct states of subjective experience that co-occur while performing some enactive skill. Keller and Landhausser \cite{Keller2012} decribe these states as: (1) reduced reflective self-consciousness, (2) modified experience of time (``time stands still"), (3) involvement and enjoyment, (4) focused concentration, (5) a strong feeling of control, and (6) the activity is perceived as rewarding in and of itself.

\subsubsection{X Learning models X}
Skill acquisition and learning across multiple domains has been proposed to follow some form of power law (Newell \& Rosenbloom, 1982).%FIXME

\subsubsection{X Learning and Eye-blink rate X}
There is a well-established relationship between attention and striatal dopamine levels, for example in studies of the `attentional blink' effect by Slagter and others \cite{Slagter2012,COLZATO2008}. Striatal dopamine can be estimated from the spontaneous eye blink rate (sEBR). Early work by Karson \cite{Karson1983} linked sEBR to striatal dopamine in humans and other primates. Taylor and colleagues \cite{Taylor1999} then localised to the caudate nucleus, whose normal function is implicated in classification learning \cite{Seger2005}; caudate nucleus volumetric asymmetry is implicated in inattention symptomatology \cite{Schrimsher2002}. Later work from Slagter \cite{Slagter2015} has shown that sEBRs may relate specifically to avoidance learning; others have linked dopamine levels and sEBR to perseverance, distractibility and cognitive control \cite{Muller2007,Dreisbach2005}.

\subsection*{Protocol and Research Questions}
\todo[author=BC,inline]{This subheader can actually be here, as was done in Ahonen etal 2018. It seems to make it easier to review if you cheat a little and add this kind of pseudo-methods section in the Intro.}
Nine subjects played a custom-designed steering game on a driving simulator for forty driving trials (duration: 2-3 mins) over eight sessions. The task is to avoid collisions with obstacles, which slow down progress, and the outcome is thus measured by duration of the trial: shorter = better. Subjects' behaviour was recorded during each trial, and they responded to the Flow Short Scale after each trial.

Our results begin by validating the novel steering task, to establish that participants do show evidence of learning in the task.

We follow up with five Research Questions:
\todo[author=BC,inline]{Add as many research questions as needed, not fixed at these 5. RQ1 might be redundant!}
\begin{enumerate}
	\item Is trial-wise Flow self-report (from hereon simply `Flow') related to performance?
	\item Is Flow related to global improvement over baseline?
	\item Is Flow related to a local performance reference, i.e. comparing to performance predicted by a learning model?
	\item Which performance reference frame has {\it more} influence on Flow: global OR local?

	\item Does striatal dopamine, as indexed by sEBR, predict learning in task performance?
	\item {\sf Sub-questions from the EBR data}?
\end{enumerate}


\section*{Results}
{\sf Up to three levels of \textbf{subheading} are permitted. Subheadings should not be numbered.}

The supplementary information describes comprehensively the performance-related data features, such as run duration; along with correlations between them.
\todo[author=BC,inline]{Add to SI section (end of paper): Table 1 and Figure 3 from LDA-C5005 report, with explanations.}

We address the validation question with evidence that participants demonstrate consistent improvement in accordance with the power law model of skill-acquisition learning. In order to investigate the possible power law behaviour, a log-log (both dependent and independent variable were log-transformed) linear model of run durations as a function of cumulative number of runs was fitted on each participant separately (a power-law curve transformed to log-space will be linear).

Figure~\ref{fig:flowcar} shows the logarithm-transformed performance of each subject in each trial. Here, the blue fit lines indicate the `ideal' learning, and the dispersion of points around this line indicates the divergence of each subject from ideal learning: points above the line indicate longer duration/worse performance than predicted by the learning curve, and vice versa.

All participant-specific models had negative slopes, which indicates that everyone improved (obtained faster run times) over time as they gained more game experience. The variation in intercepts suggests disparity in participants’ initial skill levels of the game task. The individual intercepts and slopes of the fitted log-log models are presented in Table X.
A grand model (covering all the participants) was also fitted that explained 39.6\% of variance.
\todo[author=BC,inline]{Add to SI section (end of paper): Table 5+6 from LDA-C5005 report, with explanations.}

\begin{table}[ht]
\centering
\begin{tabular}{llllllll}
\hline
Participant & Seconds & Intercept & Slope & Flow mean & Flow SD & P.I. mean & P.I. SD \\
\hline
1 & 213 & 5.36 & -0.067 & 5.10 & 0.51 & 3.53 & 0.57 \\
...&&&&&&& \\
Participant mean &&&&&&& \\
\hline
\end{tabular}
\caption{\label{tab:LCxFlow}Legend (350 words max). Example legend text.}
\end{table}
\todo[author=BC,inline]{Insert Table 2 from LDA-C5005 report, with extra column for Intercept in seconds: $e^i$ (because it's interpretable). Also add (mean+SD) columns from report Table 4, 5, 6, or 7. Also add 1 row below, with mean of all participants. This gives us 1 wide table with numerical results for Flow and learning, both participant- and group-wise.}

\begin{figure*}[!ht]
	\centering
	\includegraphics[width=\linewidth]{flowcar}
	\caption{Participant-wise data showing logarithm-transformed performance and Flow self-reports in a speeded steering task. Ordinate shows log-duration of trials, abscissa shows log-cumulative trial count. Dashed blue lines fitted to the data are `ideal' power-law learning curves, which transform to linear in log-log space.}
	\label{fig:flowcar}
\end{figure*}

Further, the points in each subplot are coloured according to Flow self-reports made after each trial, in a standardised range (i.e. original scores transformed to z-scores). The best Flow scores are green, the worst are red. Thus, we can see at a glance that the points cluster above and below the line in good agreement with the level of experienced Flow: worse performing trials (data-point above the line) tend to be red (Flow report below the subject-wise mean Flow), and vice versa.

\subsection*{RQ1}
RQ1 asks: Is Flow related to performance? Since we have established that performance improves over sessions, we can use session number as a simple proxy of performance improvement. Plotting the group-wise dispersion of Flow scores against sessions gives the result in Fig.~\ref{fig:FlowVssn}: clearly, there is no effect of session on group-wise Flow.
\todo[author=BC,inline]{Get the stats for this.}

\begin{figure*}[!ht]
	\centering
	\includegraphics[width=\nicewidth]{Flow_v_sessions}
	\caption{Boxplot representing flow in sessions 1-8. The self-report scale was 1-7.}
	\label{fig:FlowVssn}
\end{figure*}



\subsection*{RQ2}
For RQ2, we show the relation of group-wise Flow and duration scores.

\begin{figure*}[!ht]
	\centering
	\includegraphics[width=\nicewidth]{Flow_v_Dur_x_ssn}
	\caption{Duration and Flow over sessions grouped: 1, 2-4 (training), 5-8 (physiological measurements), N = 72.}
	\label{fig:FlowVdurXssn}
\end{figure*}

Pearson correlation coefficients were calculated for median duration and median flow, by the session categories above. The relationship between the variables was stronger in the later sessions, compared to the first session. In the first session, the correlation was not significant (r = -.051, p = .90, N = 9). In contrast, the Pearson correlation coefficients were significant in training sessions 2-4 (r = -.44 p < .05, N = 27) and in the sessions 5-8 (r = -.53, p < .001, N = 36). However, N for the first session was only 9.
\todo[author=BC,inline]{Get a more robust analysis, e.g. analyse session by session to equalise the N}

% FIXME: REPORT FIT OF THE LINEAR MODELS (AS A TABLE?)...
% \begin{table}[ht]
% \centering
% \begin{tabular}{|l|l|l|}
% \hline
% Condition & n & p \\
% \hline
% A & 5 & 0.1 \\
% \hline
% B & 10 & 0.01 \\
% \hline
% \end{tabular}
% \caption{\label{tab:example}Legend (350 words max). Example legend text.}
% \end{table}


\subsection*{RQ3}
For RQ3, we look at the relation between the Flow scores per trial and the value of each subject's learning curve model at that point.

For each subject, we define the agreement of the Flow scores with the local performance prediction of the learning curve model using formula~\ref{eq:flowfit}.

\begin{equation}
	\label{eq:flowfit}
	d = \sum_{x=1}^{40} \frac{f_x.(LCy_x - y_x)}{2}
\end{equation}

where $x\in X$ is a trial from the set of all trials for a subject; and at each trial $x$: $y_x$ is the observed performance, $LCy_x$ is the performance value predicted by the learning curve model, and $f_x$ is the standardised Flow score.

\todo[author=BC,inline]{Move equation declaration to Methods when finalised.}

For each participant $p \in 1..9$:
\begin{itemize}
	\item Calculate equation \ref{eq:flowfit} using observed Flow scores, to obtain $d'$
	\item For $N >= 10000$ iterations $i$:
	\begin{itemize}
		\item Randomly permute the signs of empirical Flow scores (since Flow scores are standardised the signs are uniformly distributed and can be treated as a binary label, i.e. swapped). Each resulting permutation has the same number of datapoints as the Flow measurements
		\item Calculate equation \ref{eq:flowfit} using the $n$th permutation, to obtain $d_{ip}$
	\end{itemize}
	\item Obtain $D$ the distribution (histogram) of $d$ values
	\item Compute the probability that observed value $d'$ comes from the distribution $D$
\end{itemize}

\todo[author=BC,inline]{Get effect size estimates via permutation, we have $2^{40}$ possible permutations so should obtain a good estimate of the distribution.\\
Consider other (complementary) ways to assess Flow:LC relation: e.g count of positive Flow points below the line + negative Flow above the line?}


\subsection*{RQ4}
{\sf COMPARE STRENGTH OF EFFECT FOR GLOBAL VS LOCAL RELATIONS}


\subsection*{RQ5}
sEBR shows a negative relationship to slope of the learning curve (LC) (Pearson correlation rs = -.65 , p = .06). The slope is itself negative for all subjects, which means we can say that the smaller the sEBR, the shallower the LC slope.

\begin{figure*}[!ht]
	\centering
	\includegraphics[width=\nicewidth]{EBR_v_LCslope}
	\caption{Median spontaneous eye blink rate (abscissa) and the slope of the learning curve (ordinate).}
	\label{fig:EBRvLC}
\end{figure*}

\section*{Discussion}

\paragraph{[RE FIG\ref{fig:flowcar}]}
Beyond the described results, Figure~\ref{fig:flowcar} shows some more subtle patterns. Distribution of low-to-high Flow scores tend to change with the cumulative trials for subject 3, and inversely for subject 7. This could indicate increasing and decreasing motivation, respectively.

\paragraph{[Future Work]}
Such simple trial-by-trial analyses is valuable to some degree. However it is still limited by the Flow self-report which only has one datapoint per trial. Far more powerful is to analyse {\it inside} the trials, which requires modelling of individual subject actions. %, and relating these to their physiology.

\section*{Methods}

\subsection*{Participants}
The study was conducted with nine participants recruited via student mailing lists at the University of Helsinki, as well as personal contacts (N = 9, 6 males and 3 females). The participants were between 22-38 years of age (mean = 27.1 years) with normal or corrected-to-normal visual acuity and no history of neurological or psychiatric disease. Eight of the participants had a driving license and their driving experience varied. Two participants had no or very little previous gaming experience, two participants played 1-3 hours a week, and five participants stated playing at least one hour a week. All participants were unaware of the purpose of the study and were granted 11 cultural vouchers (1 voucher is worth 5€) for participating in the study. At the time of recruiting, participants were informed that the experiment was about game experience and learning. They were told that they would get 9 vouchers for participating in all sessions and 2 extra vouchers for improving their performance in the game.

\subsection*{Design}
The experiment was divided into eight sessions on eight different days (Fig. 1). In each session, the participant played five runs of a driving game, each run lasting 2-4 min depending on their performance. After each run, the participant was shown the run duration and number of collisions, after which they filled in a self-report questionnaire (FSS). In sessions 1 and 5-8 (duration approx. an hour), physiological signals (EDA and BVP) and eye movements were measured during playing, as well as 5 minutes of baseline before playing. Sessions 2 to 4 (duration 20-30 min) served as practice sessions in which no physiological signals were recorded.

\begin{figure}[!ht]
\centering
\includegraphics[width=\nicewidth]{design}
\caption{The game was played in eight sessions on eight different days. Sessions 1 and 5 to 8 were measurement sessions where physiological signals were measured throughout the session. Sessions 2 to 4 were practice sessions. Each session consisted of five runs (2 to 4 min) followed by a self-report questionnaire (FSS, Flow Short Scale) about the latest run.}
\label{fig:design}
\end{figure}

\subsection*{Procedure}
After recruiting, participants selected eight suitable dates within a three-week period. All sessions took place between 8 a.m. and 7 p.m. at Traffic Research Unit, Department of Digital Humanities, University of Helsinki. In the first session, participants were informed about the procedure of the study and asked to fill in a questionnaire on background information, including information on health, driving experience and gaming experience. In addition, in the beginning of each session participants reported the use of contact lenses, restedness, and medicine, caffeine, and nicotine intake. In physiological measurement sessions (1 and 5 to 8), participants were then seated on a driving bench in a quiet, dimly lit room, and attached with physiological sensors and an eye-tracking headset were attached. After this, they were asked to sit still for five minutes, looking at a screen showing a dark blue colour, while the physiological baseline measurements were recorded.

The viewing distance was adjusted for each participant (so that they could place their hands on the steering wheel comfortably) and was approximately between 90 and 120 cm from the eye to the screen. The seat was located opposite the horizontal midpoint of the screen. The eye tracker headset was calibrated a minimum of three times during the session in order to detect fixation points, and additional calibrations were done if needed.

After baseline recording, the participants played five runs of the game and filled in the FSS questionnaire after each run. Before answering the FSS, participants were shown the duration of the given run, as well as the overall score with their ten best runs. In practice sessions (2 to 4) participants started playing straight after filling in the session-wise questionnaire. At the end of Session 8, the participants were debriefed and granted with the cultural vouchers.

The sessions were run by two research assistants at a time (six altogether). The room was divided so that the research assistants stayed behind a thin partition wall during playing and recording. The participants started each run by pressing a button at a self-chosen moment.

\subsection*{Materials}
\paragraph{Game.} The task in the experiment was a Python-based driving game, {\it CogCarSim}. The game code was modified to adjust starting velocity, rate of change of velocity, steering sensitivity, and graphics. In the game, the participant steered a forward-moving cube with a steering wheel to avoid obstacles on the lane. The velocity of the cube increased if no collisions occurred and slowed down if obstacles were hit: the participant was instructed to avoid as many obstacles they could in order to complete the run as fast as possible.
Data collected by CogCarSim included both run-level performance data (run duration, number of collisions, average velocity) and within-run behavioural data (steering wheel position, collisions). The positions of the cube and obstacles, and the shape and colour of obstacles were also recorded.

\paragraph{Flow Short Scale.} To measure flow experience during the game, after each run, the participants were asked to fill in the Flow Short Scale by Rheinberg, Vollmeyer and Engeser (2003). FSS is a self-report questionnaire answered shortly after fulfilling a task. It has 13 items, and it consists of the subfactors 'fluency of performance' and 'absorption by activity', both measuring the flow experience, as well as the factor 'perceived importance' measuring the importance of the given task to the participant. Since a flow state may also entail concern about failure in the task, the perceived importance factor enables controlling for the possible anxiety states. FSS also has 3 additional items on the fit of skills and demands of the task that were asked at the end of every session.

FSS was translated into Finnish by the authors because there was no translation of the scale available. The response format of FSS was a 7-point Likert scale ranging from {\it Not at all} to {\it Very much}. For the analysis, the items of fluency of performance, absorption by activity, and perceived importance were averaged. In addition, a single flow factor was created by combining factors fluency of performance and absorption by activity, as suggested by Rheinberg, Vollmeyer and Engeser (2003). Cronbach's alpha for FSS with the flow factors fluency of performance and absorption by activity (10 items) was .92. With the factor perceived importance included (13 items), Cronbach's alpha was .87.

\paragraph{Eye movements.} Eye movements were measured using Pupil Labs Binocular 120 Hz equipment with a custom-built headband. Pupil Capture software was used to collect the data from the pupil hardware.
Physiological measurements. Electrodermal activity (EDA) and blood volume pulse (BVP) were recorded at 128 Hz sampling rate using NeXus-10 (Mind Media B.V, Roermond-Herten, The Netherlands) connected to a laptop via bluetooth. The data was collected using Trusas open access software (\url{https://github.com/jampekka/trusas-nexus}).
For EDA, Ag-AgCl electrodes with 0.5 saline paste were attached to the medial side of the left foot with adhesive skin tape and gauze. The plantar site was used instead of the palmar site to minimise artefacts resulting from the use of the steering wheel, as per guidelines by Boucsein (2012).

BVP was measured using a pulse oximeter that utilizes the blood's light reflecting properties. The pulse oximeter measures the blood volume pulse by sending infrared light through the skin and measures the light reflected from blood in relation to time. The pulse oximeter sensor was attached to the left index toe of each participant.

\paragraph{Other measurements.} Background information on age, medication, eye-health, driving experience and gaming experience was collected at the beginning of the study. In addition, in the beginning of each session participants were asked information about the use of contact lenses, restedness, and medicine, caffeine and nicotine intake. During the measurement, the experimenter took notes about possible confounding factors and problems within the session.

\paragraph{Task Equipment.} Participants played CogCarSim on a 55" LG 55UF85 screen, using a Logitech G920 Driving Force steering wheel, while sitting on a Playseat Evolution Alcantara playseat. The computer % FIXME

Eye-tracking and physiology signals were recorded on an Asus UX303L laptop with Debian GNU/Linux OS.


%%%%%%%%%%%%%%%%%%%%%%%%%%%%%%%%%%%%%%%%%%%%%%%%%%%%%%%%%%%%%%%%%%%%%%%%%%%%%%%%
%%%%%%%%%%%%%%%%%%%%% POST-HOC MATERIAL OF VITAL IMPORTANCE %%%%%%%%%%%%%%%%%%%%
\wineAm{0.5}{1.0}{270}{10cm}{7cm}
\bibliography{cleanbib/CogCarFlow_bib}

\section*{Acknowledgements (not compulsory)}

Acknowledgements should be brief, and should not include thanks to anonymous referees and editors, or effusive comments. Grant or contribution numbers may be acknowledged.

\section*{Author contributions statement}

Must include all authors, identified by initials, for example:
A.A. conceived the experiment(s),  A.A. and B.A. conducted the experiment(s), C.A. and D.A. analysed the results.  All authors reviewed the manuscript.

\section*{Additional information}

The corresponding author is responsible for submitting a \href{http://www.nature.com/srep/policies/index.html#competing}{competing financial interests statement} on behalf of all authors of the paper. This statement must be included in the submitted article file.

\section*{Supplementary Information}

\end{document}
